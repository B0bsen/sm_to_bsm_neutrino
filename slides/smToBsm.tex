\documentclass[]{beamer}


\usepackage[utf8]{inputenc}
\usepackage[percent]{overpic}
\usepackage{xcolor}
\usepackage[absolute,overlay]{textpos}
\usepackage{amsmath}
\usepackage{algorithmic}
\usepackage{algorithm}
\usepackage[english]{babel}

\usepackage{graphicx}
\usepackage{paralist}
\usepackage{setspace}
\usepackage{amsfonts}
\usepackage[labelfont=bf]{caption}
\usepackage{hyperref}
\usepackage{amssymb}
\usepackage{float}
\usepackage{placeins}
\usepackage{booktabs}
\usepackage{verbatim}
\usepackage{color,array}
\usepackage{colortbl}
\usepackage{lscape}
\usepackage{url}
\usepackage{centernot}
\newcommand{\fsl}[1]{{\centernot{#1}}}
\newcommand{\defgl}{\mathrel{=\!\!\mathop:}}
\newcommand{\defgr}{\mathrel{\mathop:\!\!=}}
\def\SM{{\rm SM}}
\def\NP{{\rm NP}}
\def\ubar{\overline{u}}
\def\dbar{\overline{d}}
\def\sbar{\overline{s}}
\def\cbar{\overline{c}}
\def\bbar{\overline{b}}
\def\tbar{\overline{t}}
\def\qbar{\overline{q}}
\def\lbar{\overline{\ell}}
\def\ebar{\overline{e}}
\def\mubar{\overline{\mu}}
\def\nubar{\overline{\nu}}
\def\Qbar{\overline{Q}}
\def\Lbar{\overline{L}}
\def\ml{{{m_\ell}}}
\def\Babar{{\mbox{\slshape B\kern-0.1em{\smaller A}\kern-0.1em B\kern-0.1em{\smaller A\kern-0.2em R}}}}

\DeclareMathOperator{\Tr}{Tr}
\newcommand{\nn}{\nonumber}


\definecolor{tugreen}{HTML}{990000}
\definecolor{tugrey}{HTML}{990000}
\definecolor{tured}{HTML}{990000}
\setbeamercolor{palette compare}{bg=white!80!tugreen,fg=black}
\setbeamercolor{palette misc}{bg=white!80!tugreen,fg=black}
\setbeamercolor{palette white}{bg=white!99!black,fg=black}


\makeatletter
\renewcommand{\itemize}[1][]{%
  \beamer@ifempty{#1}{}{\def\beamer@defaultospec{#1}}%
  \ifnum \@itemdepth >2\relax\@toodeep\else
    \advance\@itemdepth\@ne
    \beamer@computepref\@itemdepth% sets \beameritemnestingprefix
    \usebeamerfont{itemize/enumerate \beameritemnestingprefix body}%
    \usebeamercolor[fg]{itemize/enumerate \beameritemnestingprefix body}%
    \usebeamertemplate{itemize/enumerate \beameritemnestingprefix body begin}%
    \list
      {\usebeamertemplate{itemize \beameritemnestingprefix item}}
      {\setlength\itemsep{0.7em}% NEW
        \def\makelabel##1{%
          {%
            \hss\llap{{%
                \usebeamerfont*{itemize \beameritemnestingprefix item}%
                \usebeamercolor[fg]{itemize \beameritemnestingprefix item}##1}}%
          }%
        }%
      }
  \fi%
  \beamer@cramped%
  \raggedright%
  \beamer@firstlineitemizeunskip%
}
\makeatother

\newcommand{\backupbegin}{
	\newcounter{finalframe}
	\setcounter{finalframe}{\value{framenumber}}
}
\newcommand{\backupend}{
	\setcounter{framenumber}{\value{finalframe}}
}


\mode<presentation>
{ \usetheme[color=green]{TU_Dortmund}
}



\title{\Large SQuIDS: A Tool to Solve Time Evolution in finite dimensional (open) Quantum Systems\\\vspace*{0.5cm}{\small An Application to Neutrino Oscillations\\\href{https://arxiv.org/abs/1412.3832.pdf}{arxiv:1412.3832}}}
\author{Dominik Hellmann}
\institute[TU Dortmund]{\scriptsize TU Dortmund\\WG Päs
}

\date{May 3, 2023}

\begin{document}
\nocite{*}


\frame{\titlepage}

\frame{\tableofcontents}

\section{Introduction}

\begin{frame}{}
  
\end{frame}

\begin{frame}{}
  
\end{frame}

\begin{frame}{}
  
\end{frame}

\begin{frame}{}
  
\end{frame}

\begin{frame}{}
  
\end{frame}

\begin{frame}{}
  
\end{frame}

\section{Installation}

\begin{frame}{Clone the Repo!}
  Git Repository inlcuding all needed files:
  \begin{block}{Instructions}
    \texttt{cd} to the location where to place the repo \\
    \texttt{git clone git@github:path/to/repo.git} \\
    \texttt{cd repo}
  \end{block}
\end{frame}

\begin{frame}{Prerequisites}
  What do we need for this tutorial?
  \begin{itemize}
    \item A unix-like (sub-)system 
    \begin{itemize}
      \item Linux
      \item Mac (+ Xcode developer tools!)
      \item On Windows: WSL
    \end{itemize}
    \item A C++ compiler
    \item Make, wget, Git
  \end{itemize}
  Use scripts \texttt{install\_gsl.sh} and \texttt{install\_SQuIDS.sh} from the repo!
\end{frame}

\section{Exercises}


\begin{frame}{Const class exercise}
  \begin{enumerate}
    \item Declare a default constructed const class object
    \item Answer the following questions:
    \begin{enumerate}
      \item How many \(\mathrm{eV}^{-1}\) correspond to \(300\,\mathrm{km}\)
      \item How many radians correspond to \(25^\circ\)
      \item If you are \(24\) years old, how many \(eV^{-1}\) are you old?
    \end{enumerate}
    \item Set the mixing parameters for three neutrino generations to:
    \begin{itemize}
      \item \(\theta_{12} = 33.48^\circ\)
      \item \(\theta_{13} = 8.55^\circ\)
      \item \(\theta_{23} = 42.3^\circ\)
    \end{itemize}
    \item Set the energy differences to:
    \begin{itemize}
      \item \(\Delta m^2_{21} = 7.5 \cdot 10^{-5} \, \mathrm{eV}^2\)
      \item \(\Delta m^2_{31} = 2.45 \cdot 10^{-3} \, \mathrm{eV}^2\)
    \end{itemize}
  \end{enumerate}
\end{frame}

\begin{frame}{SU vector exercise}
  \begin{enumerate}
    \item Declare an empty SU vector corresponding to a 3D Hilbert space
    \item Initialize an array of projectors for the three mass eigenstates (\(B_0\))
    \item Rotate them to the flavor basis (\(B_1\))
    \item Initialize a SU vector corresponding to the matrix (\(B_0\))
    \begin{align}
      \Delta \mathbb{M}^2 := \begin{pmatrix}
        0 & 0 & 0 \\
        0 & \Delta m_{21}^2 & 0 \\
        0 & 0 & m_{31}^2 \\
      \end{pmatrix}
    \end{align}
  \end{enumerate}
\end{frame}

\begin{frame}{SQuIDS application: Neutrino oscillations in vacuum}
  
\end{frame}

\begin{frame}{}
  
\end{frame}

\begin{frame}{}
  
\end{frame}

\appendix

\section{Backup}
\backupbegin

\begin{frame}
    \centering \Huge BACK UP
\end{frame}

\backupend

\end{document}
